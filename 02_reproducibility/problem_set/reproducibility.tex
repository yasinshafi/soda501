\documentclass[11pt]{article}
\usepackage{amsmath, amssymb, graphicx, hyperref}
\usepackage{enumitem}
\setlist{nosep}
\usepackage[margin=1in]{geometry}

\title{ Reproducible Research Workflows}
\author{ }
\date{ }



\begin{document}
\maketitle

\noindent \textbf{Note on data.} This assignment uses a \textbf{synthetic} voter-turnout dataset provided by \texttt{poliscitools} (via \texttt{example\_data}). The goal is to practice reproducibility tooling (dependency management, logging, packaging outputs, and replication checks)---not to draw substantive inferences about real voters.

\section*{Start Off: Verifying Your Environment}

\begin{enumerate}
\item \textbf{Environment check (required).}  
Submit proof that you successfully ran the full tutorial code on your machine. You may submit \emph{one} of the following:
\begin{itemize}
  \item A screenshot or text file showing console output that includes the printed representation shapes (document--term matrix, Word2Vec document vectors, and transformer embeddings).
  \item A screenshot of your \texttt{figures/} directory showing generated plots with timestamps.
  \item A Git commit (hash or screenshot) that includes at least one generated figure or output file.
  \item A short log file (e.g., \texttt{run\_log.txt}) containing printed diagnostics and evaluation metrics.
\end{itemize}
\end{enumerate}

\section*{Conceptual Questions}
Please write three to ten sentence explanations for each of the following questions. \textbf{You are only required to answer ONE of the two questions below.} \bigskip
 
\begin{enumerate}
\setcounter{enumi}{1}

\item Explain what problem \texttt{renv} solves in reproducible research. In your answer, describe what information is stored in \texttt{renv.lock}, what \texttt{renv::restore()} does, and why sharing code without dependency versions can fail replication even when the analysis is ``correct.''

\item Explain why logging (e.g., \texttt{logger}) is part of professional, reproducible analysis. Give two concrete examples of what you would log in a pipeline (inputs, parameters, random seeds, file paths, model summaries, warnings), and explain how logs help diagnose non-reproducible results.

  \end{enumerate}

\section*{Applied Exercises}
Use the code in the week's code tutorial and the lecture slides to answer the following questions.\bigskip

\noindent \textbf{Deliverables checklist (what your repo must contain):}
\begin{itemize}
  \item \texttt{renv.lock} (committed),
  \item your analysis script(s) (e.g., in \texttt{analysis/} or \texttt{scripts/}),
  \item \texttt{outputs/figures/} containing your plot(s),
  \item a log file (e.g., \texttt{analysis\_log.txt}) that records what ran and what files were written,
  \item a short reproducibility note (e.g., \texttt{REPRODUCE.md}) describing how to re-run your analysis from a fresh clone.
\end{itemize}

\bigskip
\begin{enumerate}
\setcounter{enumi}{3}
\item \textbf{Clone the instructor reproducibility folder and start your project.}
\begin{itemize}
  \item Clone the instructor workflow repository and work inside the \texttt{reproducibility/} folder (only that folder is relevant for this assignment).
  \item Create your own GitHub repository for submission and push your work there.
  \item Evidence (include screenshots or paste into \texttt{commands.log}):
  \begin{itemize}
    \item \texttt{git clone ...}
    \item \texttt{cd reproducibility}
    \item \texttt{git status}
  \end{itemize}
  \item Helpful tidbit:
  \begin{itemize}
    \item macOS Terminal / Windows PowerShell both support \texttt{pwd}, \texttt{ls}, and \texttt{cd}.
    \item If \texttt{git} is ``not recognized'', install Git and restart your terminal.
  \end{itemize}
\end{itemize}

\item \textbf{Reproducible workflow + three regressions + plot (Income as DV).}
\begin{itemize}
  \item \textbf{Project structure:} Ensure your folder includes (at minimum)
  \begin{itemize}
    \item \texttt{data/raw/} (store the attached CSV here),
    \item \texttt{data/processed/} (optional, only if you create derived data),
    \item \texttt{outputs/figures/} and \texttt{outputs/tables/},
    \item \texttt{analysis\_log.txt}.
  \end{itemize}

  \item \textbf{Dependency capture with \texttt{renv}:}
  \begin{itemize}
    \item Initialize or restore \texttt{renv} in your project.
    \item Create and commit \texttt{renv.lock} (use \texttt{renv::snapshot()} once your code runs).
  \end{itemize}

  \item \textbf{Run three different regressions (hard-coded, sequential) using \texttt{income} as the dependent variable:}
  \item Save a simple regression table (or clearly printed model summaries) to \texttt{outputs/tables/}.


  \item \textbf{Logging requirement:} Create/update \texttt{analysis\_log.txt} so it records:
  \begin{itemize}
    \item when the analysis was run,
    \item the number of rows loaded,
    \item which outputs were written (filenames/paths),
    \item the output of \texttt{sessionInfo()} written to a file (e.g., \texttt{outputs/session\_info.txt}).
  \end{itemize}
\end{itemize}

\item \textbf{Push your changes to your repository (submission).}
\begin{itemize}
  \item Commit and push:
  \begin{itemize}
    \item your analysis script(s),
    \item \texttt{renv.lock},
    \item \texttt{analysis\_log.txt} and \texttt{REPRODUCE.md},
    \item your output plot(s) and any small tables.
  \end{itemize}
  \item Do \emph{not} commit large intermediate files.
  \item Submission: submit the link to \textbf{your} GitHub repository.
\end{itemize}

\item \textbf{Bonus (if you finish early): Bootstrap + perfect replicability.}
\begin{itemize}
\item Run diagnostic tests on the regression and see what is driving the relationship. 
  \item Re-run the analysis using bootstrap simulations (e.g., resample rows and re-fit at least one of your models many times).
  \item Set a seed \textbf{once} at the top (use \texttt{set.seed(123)}).
  \item Save bootstrap results to \texttt{outputs/tables/} (e.g., coefficient distributions / intervals).
  \item Run the full analysis at least \textbf{twice} from a fresh R session and verify the bootstrap outputs are \textbf{identical}.
  \item In 4--6 sentences, explain what threatened replicability and what you did to eliminate it.
\end{itemize}

\end{enumerate}


\end{document}
