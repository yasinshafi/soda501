\documentclass[11pt]{article}
\usepackage{amsmath, amssymb, graphicx, hyperref}
\usepackage{enumitem}
\setlist{nosep}
\usepackage[margin=1in]{geometry}

\title{ Web Scraping in the Social Sciences}
\author{ }
\date{ }



\begin{document}
\maketitle

\section*{Start Off: Verifying Your Environment}

\begin{enumerate}
\item \textbf{Environment check (required).}  
Submit proof that you successfully ran the full tutorial code on your machine. You may submit \emph{one} of the following:
\begin{itemize}
  \item A screenshot or text file showing console output that includes the printed representation shapes (document--term matrix, Word2Vec document vectors, and transformer embeddings).
  \item A screenshot of your \texttt{figures/} directory showing generated plots with timestamps.
  \item A Git commit (hash or screenshot) that includes at least one generated figure or output file.
  \item A short log file (e.g., \texttt{run\_log.txt}) containing printed diagnostics and evaluation metrics.
\end{itemize}
\end{enumerate}

\section*{Conceptual Questions}
Please write three to ten sentence explanations for each of the following questions. \textbf{You are only required to answer ONE of the two questions below.} \bigskip
 
\begin{enumerate}
\setcounter{enumi}{1}

\item In the social sciences, what are two ethical or scientific risks of collecting data via web scraping (e.g., representativeness, privacy, terms of service, measurement error, scraping-induced missingness)? For each risk, briefly describe one practical mitigation strategy you would use in a reproducible workflow.

\item Web pages change. Explain two ways that changes to a website can break a scraping pipeline or alter the data you collect. What concrete steps would you take to (i) detect that the pipeline has broken and (ii) make the analysis replicable for someone running your code later?

  \end{enumerate}


\section*{Applied Exercises}
Use the code in the week's code tutorial and the lecture slides to answer the following questions.\bigskip

  \begin{enumerate}
\setcounter{enumi}{3}

\item Using \textbf{ten} Penn State faculty members from your department(s) or affiliated with SoDA, create a plot of \textbf{citations over time} for each professor.
\begin{itemize}
  \item Try changing the plot style (e.g., line thickness, points, theme, labels).
  \item Your figure should be readable with 10 people (facets are fine).
\end{itemize}

\item Visualize \textbf{or} discuss how the work of these professors overlaps.
\begin{itemize}
  \item One approach: use \texttt{scraped\_interests} from PSU profile pages.
  \item Another approach (advanced): compare their most common publication keywords.
\end{itemize}
\noindent Provide at least one visualization \textbf{or} a short written discussion (5--10 sentences) describing the main overlap patterns you observe.

\item What is the \textbf{median citation count (per year)} for each person in the data?
\begin{itemize}
  \item Hint: \texttt{group\_by(name)} + \texttt{summarize(median(...))}.
  \item Clearly state whether your median is computed over observed years only, or whether you treat missing years as zero (and why).
\end{itemize}

\item \textbf{Challenge Question (Optional --- if you finish early):}
Compute each scholar's \textbf{total citations} and \textbf{h-index} using \texttt{get\_profile()}, then compare those across the faculty members.
\begin{itemize}
  \item Present a clean comparison (a table and/or a bar chart).
  \item Briefly interpret what the comparison does \emph{and does not} tell you (2--4 sentences).
\end{itemize}

\end{enumerate}

\end{document}
