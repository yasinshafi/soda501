\documentclass[11pt]{article}
\usepackage{amsmath, amssymb, graphicx, hyperref}
\usepackage{enumitem}
\setlist{nosep}
\usepackage[margin=1in]{geometry}

\title{ Deterministic and Probabilistic Matching}
\author{ }
\date{ }



\begin{document}
\maketitle

\noindent \textbf{Note on data.} This problem set uses \textbf{synthetic} (simulated) individual-level data. The goal is to practice record linkage and threshold selection---not to draw substantive inferences about real people.

\section*{Start Off: Verifying Your Environment}

\begin{enumerate}
\item \textbf{Environment check (required).}  
Submit proof that you successfully ran the full tutorial code on your machine. You may submit \emph{one} of the following:
\begin{itemize}
  \item A screenshot or text file showing console output that includes the printed representation shapes (document--term matrix, Word2Vec document vectors, and transformer embeddings).
  \item A screenshot of your \texttt{figures/} directory showing generated plots with timestamps.
  \item A Git commit (hash or screenshot) that includes at least one generated figure or output file.
  \item A short log file (e.g., \texttt{run\_log.txt}) containing printed diagnostics and evaluation metrics.
\end{itemize}
\end{enumerate}

\section*{Conceptual Questions}
Please write three to ten sentence explanations for each of the following questions. \textbf{You are only required to answer ONE of the two questions below.} \bigskip
 
\begin{enumerate}
\setcounter{enumi}{1}

\item Explain why deterministic (exact) matching can fail even when two datasets contain the same underlying individuals. In your answer, define the trade-off between \textbf{false matches} and \textbf{missed matches}, and explain how probabilistic matching (e.g., \texttt{fastLink}) attempts to manage that trade-off.

\item Matching errors are often \textbf{not random}. Give two reasons why record linkage error might vary across individuals or groups (e.g., name commonness, transliteration, data-entry error, moving/ZIP changes). Explain one implication for social science inference if linkage quality differs systematically across groups.

  \end{enumerate}


\section*{Applied Exercises}
Use the code in the week's code tutorial and the lecture slides to answer the following questions.\bigskip

  \begin{enumerate}
\setcounter{enumi}{3}

\item \textbf{Generate (or load) the synthetic data + deterministic matching.}
Run the provided script to generate and save \texttt{dataset\_a.csv} and \texttt{dataset\_b.csv}, then load them into R.
\begin{itemize}
  \item Perform deterministic (exact) matching on \texttt{firstname}, \texttt{lastname}, \texttt{birthyear}, and \texttt{zipcode} (e.g., \texttt{merge(..., by = ...)}).
  \item Report the number of deterministic matches and the match rate (matches divided by \texttt{nrow(df\_a)}).
  \item In 3--6 sentences, explain why the deterministic match count looks the way it does in this simulation.
\end{itemize}

\item \textbf{Probabilistic matching with \texttt{fastLink} + threshold curve.}
Using \texttt{fastLink}, match \texttt{df\_a} and \texttt{df\_b} on \texttt{firstname}, \texttt{lastname}, \texttt{birthyear}, and \texttt{zipcode}.
\begin{itemize}
  \item Use \texttt{fastLink(..., return.all = TRUE)}.
  \item Use \texttt{getMatches(..., threshold.match = t)} for a grid of thresholds from 0 to 1 (e.g., increments of 0.01).
  \item Create a plot of \textbf{number of matches} vs.\ \textbf{threshold}.
  \item In 4--6 sentences, describe how and why the curve changes as the threshold increases.
\end{itemize}

\item \textbf{Match quality, choosing a threshold, and interpreting posteriors.}
Using the probabilistic matches, evaluate match quality as the threshold changes and justify a final choice.
\begin{itemize}
  \item Create a low-threshold ``candidate match'' set (e.g., \texttt{threshold.match = 0.000001}) and then group matches by posterior bins (e.g., 0.0--0.1, 0.1--0.2, \dots, 0.9--1.0).
  \item For each posterior bin, compute:
  \begin{enumerate}
    \item Levenshtein distance for first names (e.g., \texttt{stringdist(..., method = "lv")}),
    \item Levenshtein distance for last names,
    \item absolute difference in birth year.
  \end{enumerate}
  \item Make at least one plot that shows how these distances relate to posterior scores (e.g., boxplots of distance by posterior bin).
  \item Based on your diagnostics, choose a threshold you would use for this dataset and defend your choice (5--8 sentences).
  \item In your discussion, address:
  \begin{enumerate}
    \item how deterministic vs probabilistic matching differ in the number of matches found,
    \item how changing the threshold affects both the number of matches and match quality,
    \item at least two limitations or biases of each approach,
    \item the relationship between string distance and the posterior/threshold measure.
  \end{enumerate}
\end{itemize}

\item \textbf{Challenge Question (Optional --- if you finish early):}
Experiment with modelling choices in \texttt{fastLink}.
\begin{itemize}
  \item Try at least \textbf{two} different matching-variable sets (e.g., drop \texttt{zipcode}; or match only on names + birthyear).
  \item Try at least \textbf{two} different string distance methods for names (see \texttt{stringdist} methods such as Jaro-Winkler vs Levenshtein).
  \item For each configuration, plot the threshold--matches curve and summarize how match quality changes (5--8 sentences).
  \item Briefly discuss the implications of these modelling choices for real social-science linkage tasks (e.g., voter files, administrative records, platform data).
\end{itemize}

\end{enumerate}

\end{document}
