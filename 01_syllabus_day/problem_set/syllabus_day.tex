\documentclass[11pt]{article}
\usepackage{amsmath, amssymb, graphicx, hyperref}
\usepackage{enumitem}
\usepackage{hyperref}
\setlist{nosep}
\usepackage[margin=1in]{geometry}

\title{Introduction to Big Data}
\author{ }
\date{ }

\begin{document}
\maketitle

\section*{Course setup}

\noindent Upload a single PDF that includes screenshots confirming completion of the items below. 
Screenshots must show your name/username where applicable.

\begin{enumerate}
  \item \textbf{GitHub account}
  \begin{enumerate}
    \item Create a GitHub account.
    \item Screenshot: your GitHub profile page (showing your username).
  \end{enumerate}

  \item \textbf{Git installed (no Git assignment yet)}
  \begin{enumerate}
    \item Install Git on your computer.
    \item Screenshot: a terminal/PowerShell window showing \texttt{git --version}.
  \end{enumerate}

  \item \textbf{Python installed and accessible from the command line}
  \begin{enumerate}
    \item Install Python (version 3.10+).
    \item Screenshot: a terminal/PowerShell window showing \texttt{python --version}.
  \end{enumerate}

  \item \textbf{R installed}
  \begin{enumerate}
    \item Install R (version 4.3+).
    \item Screenshot: your R console showing \texttt{R.version.string}.
  \end{enumerate}

  \item \textbf{IDE installed (choose at least one)}
  \begin{enumerate}
    \item Install at least one of the following: \textbf{RStudio}, \textbf{VS Code}, or \textbf{PyCharm (Community Edition is fine)}.
    \item Screenshot: the IDE open on your computer.
  \end{enumerate}

  \item \textbf{Cloud-synced course folder}
  \begin{enumerate}
    \item Create a folder for this course that is backed up and synced (Dropbox, Google Drive, OneDrive, etc.).
    \item Screenshot: your file explorer showing the folder and the sync indicator (if available).
  \end{enumerate}

  \item \textbf{Compression utility}
  \begin{enumerate}
    \item Ensure you can open \texttt{.zip} and \texttt{.gz}/\texttt{.tar.gz} files.
    \item Windows users: install 7-Zip (or an equivalent tool).
    \item Screenshot: a file explorer view showing a \texttt{.zip} (or \texttt{.tar.gz}) file opened/extracted.
  \end{enumerate}

  \item \textbf{Jupyter installed (Notebook or Lab)}
  \begin{enumerate}
    \item Install Jupyter Notebook or JupyterLab.
    \item Screenshot: a browser window showing Jupyter running (home page listing files is sufficient).
  \end{enumerate}

  \item \textbf{SQLite tools installed}
  \begin{enumerate}
    \item Install SQLite.
    \item Install a SQLite database browser (e.g., DB Browser for SQLite).
    \item Screenshot: the database browser open (and, if possible, a \texttt{.db} file opened).
  \end{enumerate}

  \item \textbf{IRB training (if you are not already certified)}
  \begin{enumerate}
    \item Complete the required human-subjects/IRB training for your program/university.
    \item Screenshot: completion certificate or training record.
  \end{enumerate}
\end{enumerate}

\section*{Interests / sign-up}

\noindent During the semester, you will: (a) work on a group project, (b) lead a coding demonstration, and (c) lead a group discussion for the articles. \href{https://docs.google.com/spreadsheets/d/1w-HvzY-aey1HhAOc2fU-II7EfJXMe9tO5QFb6u4i1h8/edit?usp=sharing}{Click this sentence to be redirected to the course sign-up google sheet.} \bigskip

\begin{enumerate}
  \setcounter{enumi}{10}
  \item On \textbf{Sheet 1}, list the week(s) of material you are most interested in.
  \item On \textbf{Sheet 2}, select \textbf{one week} that you will lead the \textbf{coding demonstration}. It cannot be the same week you lead the article discussion.
  \item On \textbf{Sheet 3}, select \textbf{one week} that you will lead the \textbf{article discussion}. It cannot be the same week you lead the coding demonstration.
\end{enumerate}

\end{document}
