\documentclass[11pt]{article}
\usepackage{amsmath, amssymb, graphicx, hyperref}
\usepackage{enumitem}
\setlist{nosep}
\usepackage[margin=1in]{geometry}

\title{ Time Series Analyses for Social Data}
\author{ }
\date{ }

\begin{document}
\maketitle


\section*{Start Off: Verifying Your Environment}

\begin{enumerate}
\item \textbf{Environment check (required).}  
Submit proof that you successfully ran the full tutorial code on your machine. You may submit \emph{one} of the following:
\begin{itemize}
  \item A screenshot or text file showing console output that includes the printed representation shapes (document--term matrix, Word2Vec document vectors, and transformer embeddings).
  \item A screenshot of your \texttt{figures/} directory showing generated plots with timestamps.
  \item A Git commit (hash or screenshot) that includes at least one generated figure or output file.
  \item A short log file (e.g., \texttt{run\_log.txt}) containing printed diagnostics and evaluation metrics.
\end{itemize}
\end{enumerate}

\section*{Conceptual Questions}
Please write a three to ten sentence explanation for \textbf{ONE} of the two questions below.

\begin{enumerate}
\setcounter{enumi}{1}
\item \textbf{Time leakage and honest evaluation.}
In the coding tutorial, you computed RMSE under (i) a random train/test split and (ii) a time-based split (train on past, test on future).
\begin{itemize}
  \item Explain what \emph{time leakage} is and why random splits usually inflate model performance for time-indexed data.
  \item Explain why ``train on past, test on future'' is the default evaluation rule.
  \item Describe what \emph{rolling-origin} (backtesting) adds beyond a single time split.
\end{itemize}

\item \textbf{Causal timing designs and placebos.}
Interrupted Time Series (ITS) is often used when an intervention occurs at a known time $t_0$ but randomization is not available.
\begin{itemize}
  \item State one key identification assumption behind ITS.
  \item Explain the difference between a ``level change'' and a ``trend change'' at $t_0$.
  \item Propose one placebo diagnostic and explain what failure would look like (e.g., a fake intervention date, pre-trend check).
\end{itemize}
\end{enumerate}

\section*{Applied Exercises}
Use the code in \texttt{timeseries.R} and the lecture slides to answer the following questions.\\bigskip

\begin{enumerate}
\setcounter{enumi}{3}

\item \textbf{Decomposition: trend + seasonality + residual (save a figure).}
Extend the synthetic daily time series in \texttt{timeseries.R} by decomposing it into components.
\begin{itemize}
  \item Convert the daily series to a \texttt{ts} object with weekly frequency (\texttt{frequency = 7}).
  \item Run a seasonal-trend decomposition (e.g., \texttt{stl()}).
  \item Create a single figure showing the observed series and the decomposed components.
  \item Save the figure as \texttt{outputs/figures/decomposition.png}.
\end{itemize}

\item \textbf{Rolling-origin evaluation (backtesting) for forecasting RMSE.}
The lecture emphasizes that evaluation should mimic deployment (fit on past, predict future).
Implement a rolling-origin backtest that produces \emph{many} out-of-sample errors instead of a single split.
\begin{itemize}
  \item Choose a forecast horizon $h = 1$ day ahead.
  \item Choose an initial training window (e.g., first 300 days).
  \item For each time $t$ from the end of the initial window to the end of the series:
    \begin{enumerate}
      \item fit an AR(1) model to data up to $t$ (use \texttt{arima(..., order=c(1,0,0))} or \texttt{forecast::auto.arima} if available),
      \item forecast $y_{t+1}$,
      \item store the one-step-ahead error $e_{t+1} = y_{t+1} - \hat{y}_{t+1}$.
    \end{enumerate}
  \item Compute RMSE across the backtest errors and compare it to the single time-split RMSE from the tutorial.
  \item Save:
    \begin{itemize}
      \item a CSV of the backtest errors (\texttt{date}, \texttt{y}, \texttt{yhat}, \texttt{error}) as \texttt{outputs/tables/backtest\_errors.csv}, and
      \item a line plot of $y_t$ and $\hat{y}_t$ over the test region as \texttt{outputs/figures/backtest\_forecast.png}.
    \end{itemize}
\end{itemize}

\item \textbf{Interrupted Time Series (ITS): level and slope change + placebo date.}
Create a synthetic intervention at time $t_0$ on top of a trend (use the lecture’s ITS framing).
\begin{itemize}
  \item Pick an intervention date $t_0$ around the middle of the series (report your choice).
  \item Create an intervention indicator $I[t \ge t_0]$ and a post-intervention time counter $(t - t_0)\,I[t \ge t_0]$.
  \item Fit an ITS regression:
  \[
    y_t = \alpha + \delta t + \tau_1 I[t \ge t_0] + \tau_2 (t - t_0)I[t \ge t_0] + \varepsilon_t
  \]
  \item Plot three lines on the same figure:
    \begin{itemize}
      \item observed $y_t$,
      \item fitted values from the ITS model,
      \item counterfactual values setting $\tau_1=\tau_2=0$ (pre-period trend extended forward).
    \end{itemize}
  \item Run a placebo ITS with a \emph{fake} intervention date in the pre-period and report whether it produces a large ``effect.'' (Briefly interpret.)
  \item Save:
    \begin{itemize}
      \item coefficient table for the real ITS and placebo ITS as \texttt{outputs/tables/its\_results.csv}, and
      \item your main ITS figure as \texttt{outputs/figures/its\_plot.png}.
    \end{itemize}
\end{itemize}

\end{enumerate}

\section*{Challenge Question (Optional --- if you finish early)}
Choose \textbf{ONE} option.

\begin{enumerate}[label=(\alph*)]
\item \textbf{Switchback experiment (time-block randomization).}
Simulate a switchback design where treatment alternates by time blocks (e.g., every 7 days) and estimate the effect.
\begin{itemize}
  \item Create a block assignment over time (alternating 0/1).
  \item Generate an outcome with trend + noise + a treatment effect that operates only during treated blocks.
  \item Estimate the difference in means across treated vs control blocks.
  \item Placebo: shift assignment by 1 block (lag the assignment) and show the estimated effect shrinks toward 0.
  \item Save a plot of the time series with shaded treated blocks as \texttt{outputs/figures/switchback.png}.
\end{itemize}

\item \textbf{Sequential testing / peeking inflation (simulation).}
Simulate repeated monitoring where you compute a p-value every day after a minimum start date.
\begin{itemize}
  \item Under a true null (no treatment effect), compute daily ``p-values'' for a naive treated vs control comparison.
  \item Show that the probability of getting at least one p-value $< 0.05$ is much larger than 0.05.
  \item Briefly explain why this happens and name one control strategy (alpha spending, pre-specified stopping).
\end{itemize}
\end{enumerate}

\end{document}
